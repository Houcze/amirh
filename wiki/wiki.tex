\documentclass{article}
\usepackage{geometry}
\usepackage{bbm}
\usepackage{amsfonts,amssymb}
\usepackage{amsmath}
\usepackage{bbm}
\usepackage{setspace}
\usepackage{fontspec}
\usepackage{caption}
\usepackage{graphicx, subfig}

\renewcommand{\baselinestretch}{1.0}
\setlength{\baselineskip}{22pt}
\setmainfont{Source Code Pro}
\geometry{a4paper, left=2cm, right=2cm, top=1cm, bottom=1cm}

\title{AMIRH Document}
\author{HOU Chengze}
\date{2022}
\begin{document}
\begin{sloppypar}
\maketitle
The AMIRH project aims to reproduce the complex atmospheric phenomena in the atmosphere of terrestrial planets.
It considers the radiation from the universe and the properties of the planet itself.
This model does not take a specific planet as the hypothetical object. I look forward to exploring the relationship between the atmosphere and the planet.
With the powerful computing power provided by the GPU, the model can support a very detailed resolution.

\section{Where I start}
The original idea for this project came from the barotropic atmospheric model
proposed by Chung-Ying Hu in 1975. His model is based on,\newline
\begin{itemize}
    \item [*] Equation of motion
    \item [*] Equation of continuity
    \item [*] Equation of gas state
\end{itemize}
And these assumptions,\newline
\begin{itemize}
    \item [*] Non-divergent
    \item [*] Incompressible
    \item [*] No surface friction
    \item [*] Baroclinic effect
\end{itemize}
The master equation is,
\begin{equation}
    \frac{\partial}{\partial{t}}\nabla^{2}z=J(f+\zeta, z)
\end{equation}
$z$ is the air pressure at the isobaric surface, $f$ is Coriolis parameter, $J(A, B)=\frac{\partial A}{x}\frac{\partial B}{y}-\frac{\partial A}{y}\frac{\partial B}{x}$. By applying the finite difference and the relaxation
method:\newline
\begin{equation}
    (\frac{\partial{z}}{\partial{t}})^{n+1}_{i,j} =-\frac{1}{4} [ (\frac{\partial{z}}{\partial{t}})_{i+1,j}+ (\frac{\partial{z}}{\partial{t}})_{i-1,j}+ (\frac{\partial{z}}{\partial{t}})_{i,j+1}+ (\frac{\partial{z}}{\partial{t}})_{i,j-1}- \frac{d^2}{m^2}J(f+\zeta, z)_{i,j} ]^n
\end{equation}
\noindent Repeatedly calculate (2) until,\newline
\begin{equation}
    \max\lvert(\frac{\partial{z}}{\partial{t}})^{n+10}-(\frac{\partial{z}}{\partial{t}})^{n}\rvert\leq\varepsilon
\end{equation}
$d$ is the real distance corresponding to the grid points, $\varepsilon$ is the lowest value for convergence, which Hu sets to $1.0\times10^{-6}$ m/sec. If $t=0$, by applying the Time-forward method, we could obtain the next $z$,
when $t>0$ by applying the Center-difference method:

\begin{equation}
    \begin{cases}
        z^{t+\Delta{t}}_{i,j}=z^{t}_{i,j}+\Delta{t}(\frac{\partial{z}}{\partial{t}})^{t}_{i,j}            & t=0 \\
        z^{t+\Delta{t}}_{i,j}=z^{t-\Delta{t}}_{i,j}+2\Delta{t}(\frac{\partial{z}}{\partial{t}})^{t}_{i,j} & t>0 \\
    \end{cases}
\end{equation}
The boundary condition is Free slip boundary condition, which assumes that the atmosphere can flow freely at the boundary without being affected by frictional resistance, \newline
\begin{equation}
    \begin{cases}
        \frac{\partial v}{\partial x}=0 \\ \frac{\partial u}{\partial y}=0 \\
    \end{cases}
\end{equation}
Whihc could be expressed as, \newline
\begin{equation}
    z_{outside the boundary}-boundary = 2z_{boundary} - z_{inside the boundary}-boundary
\end{equation}
The initial condition is by applying several time slots of observations.
The $J$ above is solved by 12-points approximation, Fig.1 is the corresponding computation domain and grid system. \newline

\begin{figure*}
    \centering
    \includegraphics[scale=0.45]{../imgs/wiki/12-points-fd.png}
    \caption{computation domain and grid system}
    \label{fig:1}
\end{figure*}
The law of conservation of energy is verified.
\begin{equation}
    J(A, B) = 2J_1(A, B)-J2(A, B)
\end{equation}
\begin{equation}
    J_1(A, B) = \frac{1}{3}[J^{++}(A, B)+J^{+\times}(A, B)+J^{\times+}(A, B)]
\end{equation}
\begin{equation}
    J^{++}(A, B) = \frac{1}{4d^2}[(A_1-A_3)(B_3-B_4)-(A_2-A_4)(B_1-B_3)]
\end{equation}
\begin{equation}
    J^{+\times}(A, B)= \frac{1}{4d^2}[A_5(B_2-B_1)-A_7(B_3-B_4)+A_6(B_3-B_2)-A_8(B_4-B_1)]
\end{equation}
\begin{equation}
    J^{\times+}(A, B) = \frac{1}{4d^2}[A_1(B_5-B_8)-A_3(B_6-B_7)-A_2(B_5-B_6)+A_4(B_8-B_7)]
\end{equation}

\begin{equation}
    J_2(A, B) = \frac{1}{3}[J^{\times\times}(A, B)+J^{\times+}(A, B)+J^{+\times}(A, B)]
\end{equation}
\begin{equation}
    J^{\times\times}(A, B) = \frac{1}{8d^2}[(A_5-A_7)(B_6-B_8)-(A_6-A_8)(B_5-B_7)]
\end{equation}
\begin{equation}
    J^{+\times}(A, B)= \frac{1}{8d^2}[A_5(B_9-B_{12})-A_7(B_{10}-B_4)-A_6(B_9-B_{10})+A_8(B_{12}-B_{11})]
\end{equation}
\begin{equation}
    J^{\times+}(A, B) = \frac{1}{8d^2}[A_9(B_6-B_5)-A_{11}(B_7-B_8)+A_{10}(B_7-B_6)-A_{12}(B_8-B_5)]
\end{equation}
Here $A_i, B_i (i=1,......12)$ is the corresponding value of A, B at $i$-th points.
\section{3D model}
As for the 3D model, the motion equations are,
\begin{equation}
    \begin{cases}
        \frac{du}{dt}=\frac{uv\tan\phi}{r}-\frac{uw}{r}-\frac{1}{\rho}\frac{\partial p}{r\cos\phi\partial\lambda}+fv-\hat{f}\omega+F_{\lambda}\frac{d}{dt}+\frac{u^2}{\tan\phi}{r}
        \\ \frac{dv}{dt}=-\frac{u^2\tan\phi}{r}-\frac{vw}{r}-\frac{1}{\rho}\frac{\partial p}{r\partial \phi}-fu+F_{\phi} \\
        \frac{d\omega}{dt}=\frac{u^2+v^2}{r}-\frac{1}{\rho}\frac{\partial p}{\partial r}-g+\hat{f}u+F_r \\
    \end{cases}
\end{equation}
\subsection{sigma vertial coordinate}
A vertical coordinate for atmospheric models defined as pressure normalized by
its surface value, or as the difference in pressure and its value at the top of
the model atmosphere normalized by the surface value of this difference.
\newline Thus,
\begin{equation}
    \sigma=\frac{p-p_T}{p_S-p_T}
\end{equation}

\begin{figure*}
    \centering
    \includegraphics[scale=0.45]{../imgs/wiki/sigma_diff.JPG}
    \caption{Before sigma and sigma}
    \label{fig:2}
\end{figure*}
The difference could be seen in figure 2.
\subsection{Hybrid sigma}
The hybrid $\sigma$ coordinate remodified the high pressure level.
\begin{equation}
    p_d=B(\eta)(p_s-p_t)+[\eta-B(\eta)](p_0-p_t)+p_t
\end{equation}
where $p_0$ is a reference sea-level pressure. Here, $B(\eta)$ defines the relative weighting between the terrain-coordinate for $B(\eta)=\eta$ and
reverts to a hydrostatic pressure coordinate for $B(\eta)=0$. To smoothly transition from a sigma near the surface to a pressure coordinate at upper levels,
$B(\eta)$ is defined by a third order polynomial \newline
\begin{equation}
    B(\eta)=c_1+c_2\eta+c_3\eta^{2}+c_4\eta^{3}
\end{equation}
By applying the boundary conditions, there should be,
\begin{equation}
    \begin{cases}
        B(1)=1             \\
        B_{\eta}=1         \\
        B(\eta_c)=0        \\
        B_{\eta}(\eta_c)=0 \\
    \end{cases}
\end{equation}
such that,
\begin{equation}
    \begin{cases}
        c_1=\frac{2\eta_c^2}{(1-\eta_c)^3}                  \\
        c_2=\frac{-\eta_c(4+\eta_c+\eta_c^2)}{(1-\eta_c)^3} \\
        c_3=\frac{2\eta_c(1+\eta_c+\eta_c^2)}{(1-\eta_c)^3} \\
        c_4=\frac{-(1+\eta_c)}{(1-\eta_c)^3}                \\
    \end{cases}
\end{equation}
where $\eta_c$ is a specified value of $\eta$ at which it becomes a pure pressure coordinate.
The difference could be seen in figure 3.
\begin{figure*}
    \centering
    \includegraphics[scale=0.45]{../imgs/wiki/hybridsigma.png}
    \caption{sigma and hybrid sigma}
    \label{fig:3}
\end{figure*}
\newline
What should the $\eta_c$ be for Mars? \newline
The vertical coordinate metric is defined as
\begin{equation}
    \mu_d=\frac{\partial{p_d}}{\partial{\mu}}=B_{\mu}(\mu)(p_s-p_t)+[1-B_{\mu}(\mu)](p_0-p_t)
\end{equation}
Since $\mu_d\delta\mu=\delta p_d=-g\rho\delta{z}$ is proportional to the mass per unit area weightin
a gird cell, the appripirate flux forms for the prognostic variables are defined as
\begin{equation}
    \begin{cases}
        \boldsymbol{{\rm V}}={\mu_d}\boldsymbol{{\rm v}}=(U,V,W)       \\
        \Omega=\mu_d\omega     \\
        \Theta_m=\mu_d\theta_m \\
        Q_m={\mu_d}{q_m}
    \end{cases}
\end{equation}
Here, $\boldsymbol{{\rm v}}=(u,v,\omega)$ are the covariant velocities in the horizontal and vertical directions, while $\omega=\dot{\mu}$ is
the contravariant `vertical' velocity. $\theta_m=\theta(1+(\frac{R_v}{R_d})q_v)\approx\theta(1+1.61q_v)$ is the moist potential
temperature and $Q_m=\mu{dq_m}$, where $q_m=q_v$,$q_c$,$q_r$$\ldots$
represnets the mixing ratio of moisture variables (water vapor, cloud water, rain water,$\ldots$).
Although the geopotential $\phi=gz$ is also a prognostic variable in the governing equations of the model, it is not written in flux-form 
as $\mu_d\phi$ is not a conserved quantity.

% 是否需要保留本段
\subsection{Coupled Algorithm}
The pressure based solver allows us to solve the problem in either a segregated or coupled manner. Using the coupled approach offers 
some advantages over the non-coupled or segregated approach. The coupled scheme obtains a robust and efficient single phase implementation
for steady-state flows, with superior performance compared to the segregated solution schemes. This pressure-based coupled algorithm
offers an alternative to the density-based and pressure-based segregated algorithm with SIMPLE-type pressure-velocity coupling. For 
transient flows, using the coupled algorithm is necessary when the quality of the mesh is poor, or if large time steps are used.

The pressure-based segregated algorithm solves the momentum equation and pressure correction equations separately. 
This semi-implicit solution method results in slow convergence.

\subsection{Flux-Form Euler Equations Under Sphere Coordinate}
Using the variables defined above, the flux-form Euler equations can be written as 
%% 1    0                       0
%% 0    r^2                     0
%% 0    0                       r^2\sin^2\theta           
\begin{equation}
    \begin{split}
    \partial_t{U}+
    (\frac{1}{r^2}\frac{\partial}{\partial{r}}(r^2Uu)+\frac{1}{r\sin\theta}\frac{\partial}{\partial\theta}(\sin\theta{Vu})+\frac{1}{r\sin\theta}\frac{\partial{Wu}}{\partial{\psi}})+ \\
    {\mu_d}\alpha(\sin\theta\cos\psi\frac{\partial{p}}{\partial{r}}+\frac{\cos\theta\cos\psi}{r}\frac{\partial{p}}{\partial{\theta}}-\frac{\sin\psi}{r\sin\theta}\frac{\partial{p}}{\partial{\psi}})+ \\
    (\frac{\alpha}{\alpha_d}\partial_{\mu}p(\sin\theta\cos\psi\frac{\partial{\phi}}{\partial{r}}+\frac{\cos\theta\cos\psi}{r}\frac{\partial{\phi}}{\partial{\theta}}-\frac{\sin\psi}{r\sin\theta}\frac{\partial{\phi}}{\partial{\psi}})) \\
    =F_{U}
    \end{split}
\end{equation}
\begin{equation}
    \begin{split}
    \partial_t{V}+
    (\frac{1}{r^2}\frac{\partial}{\partial{r}}(r^2Uv)+\frac{1}{r\sin\theta}\frac{\partial}{\partial\theta}(\sin\theta{Vv})+\frac{1}{r\sin\theta}\frac{\partial{Wv}}{\partial{\psi}})+ \\
    {\mu_d}\alpha(\sin\theta\sin\psi\frac{\partial{p}}{\partial{r}}+\frac{\cos\theta\sin\psi}{r}\frac{\partial{p}}{\partial{\theta}}+\frac{\cos\psi}{r\sin\theta}\frac{\partial{p}}{\partial{\psi}})+ \\
    (\frac{\alpha}{\alpha_d}\partial_{\mu}p(\sin\theta\cos\psi\frac{\partial{\phi}}{\partial{r}}+\frac{\cos\theta\cos\psi}{r}\frac{\partial{\phi}}{\partial{\theta}}-\frac{\sin\psi}{r\sin\theta}\frac{\partial{\phi}}{\partial{\psi}})) \\
    =F_{V}
    \end{split}
\end{equation}
\begin{equation}
    \begin{split}
    \partial_t{W}+
    (\frac{1}{r^2}\frac{\partial}{\partial{r}}(r^2U\omega)+\frac{1}{r\sin\theta}\frac{\partial}{\partial\theta}(\sin\theta{V\omega})+\frac{1}{r\sin\theta}\frac{\partial{W\omega}}{\partial{\psi}})+ \\
    -g[\frac{\alpha}{\alpha_{d}}\partial_{\eta}p-\mu_{d}]
    =F_{W}
    \end{split}
\end{equation}
\begin{equation}
    \partial_t{\Theta_m}+(\frac{1}{r^2}\frac{\partial}{\partial{r}}(r^2U\theta_m)+\frac{1}{r\sin\theta}\frac{\partial}{\partial\theta}(\sin\theta{V\theta_m})+\frac{1}{r\sin\theta}\frac{\partial{W\theta_m}}{\partial{\psi}})=F_{\Theta_m}
\end{equation}
\begin{equation}
    \partial_t{\mu_d}+(\frac{1}{r^2}\frac{\partial}{\partial{r}}(r^2U)+\frac{1}{r\sin\theta}\frac{\partial}{\partial\theta}(\sin\theta{V})+\frac{1}{r\sin\theta}\frac{\partial{W}}{\partial{\psi}})=0
\end{equation}
\begin{equation}
    \partial_t{\phi}+\mu_d^{-1}[U\frac{1}{r^2}\frac{\partial}{\partial{r}}(r^2U\phi)+V\frac{1}{r\sin\theta}\frac{\partial}{\partial\theta}(\sin\theta{V\phi})+W\frac{1}{r\sin\theta}\frac{\partial{W\phi}}{\partial{\psi}}-gW]=0
\end{equation}
\begin{equation}
    \partial_t{Q_m}+(\frac{1}{r^2}\frac{\partial}{\partial{r}}(r^2Uq_m)+\frac{1}{r\sin\theta}\frac{\partial}{\partial\theta}(\sin\theta{Vq_m})+\frac{1}{r\sin\theta}\frac{\partial{Wq_m}}{\partial{\psi}})=F_{Q_m}
\end{equation}
with the diagnostic equation for dry hydrostatic pressure\newline
\begin{equation}
    \partial_{\eta}\phi=-\alpha_d\mu_d
\end{equation}
and the diagnostic relation for the full pressure (dry air plus water vapor)\newline
\begin{equation}
    p=p_0(\frac{R_d\theta_m}{p_0\alpha_d})^{\gamma}
\end{equation}
In these equations, $\alpha_d$ is the inverse density of the dry air $\frac{1}{\rho_d}$ and $\alpha$ is the inverse density taking into account the full parcel density $\alpha=\alpha_d(1+q_v+q_c+q_r+q_i+\ldots)^{-1}$.
$\gamma=\frac{c_p}{c_v}=1.4$ is the ratio of the heat capacities for dry air, $R_d$ is the gas constant for dry air, and $p_0$ is a refernce surface pressure (On earth typically $10^5$ Pascals). The right-hand-side (RHS) terms
$F_U$, $F_V$, $F_W$, and $F_{\Theta}$ represent forcing terms arising from model physics, turbulent mixing, spherical projections, and the planet's rotation.\newline
\subsection{Map Projections, Coriolis and Curvature Terms}
Map scale factors $m_x$ and $m_y$ are defined as the ratio of the distance in computational space to the corresponding distance on the planet's surface:
\begin{equation}
    (m_x,m_y)=\frac{(\delta{x}, \delta{y})}{\text{distance on the planet}}
\end{equation}
Thus we could refine the variables as
\begin{equation}
    \begin{cases}
        U=\frac{\mu_d{u}}{m_y}\\
        V=\frac{\mu_d{v}}{m_x}\\
        W=\frac{\mu_d{w}}{m_y}\\
        \Omega=\frac{\mu_d{u}}{m_y}
    \end{cases}
\end{equation}
By applying these redifined variables, the governing prognostic equations including map factors could be written as\newline
\begin{equation}
    \begin{split}
    \partial_t{U}+
    m_x(\frac{1}{r^2}\frac{\partial}{\partial{r}}(r^2Uu)+\frac{1}{r\sin\theta}\frac{\partial}{\partial\theta}(\sin\theta{Vu}))+\frac{1}{r\sin\theta}\frac{\partial{Wu}}{\partial{\psi}}+ \\
    \frac{m_x}{m_y}{\mu_d}\alpha(\sin\theta\cos\psi\frac{\partial{p}}{\partial{r}}+\frac{\cos\theta\cos\psi}{r}\frac{\partial{p}}{\partial{\theta}}-\frac{\sin\psi}{r\sin\theta}\frac{\partial{p}}{\partial{\psi}})+ \\
    \frac{m_x}{m_y}(\frac{\alpha}{\alpha_d}\partial_{\mu}p(\sin\theta\cos\psi\frac{\partial{\phi}}{\partial{r}}+\frac{\cos\theta\cos\psi}{r}\frac{\partial{\phi}}{\partial{\theta}}-\frac{\sin\psi}{r\sin\theta}\frac{\partial{\phi}}{\partial{\psi}})) \\
    =F_{U}
    \end{split}
\end{equation}
\begin{equation}
    \begin{split}
    \partial_t{V}+
    m_y(\frac{1}{r^2}\frac{\partial}{\partial{r}}(r^2Uv)+\frac{1}{r\sin\theta}\frac{\partial}{\partial\theta}(\sin\theta{Vv}))+\frac{m_y}{m_x}\frac{1}{r\sin\theta}\frac{\partial{Wv}}{\partial{\psi}}+ \\
    \frac{m_y}{m_x}{\mu_d}\alpha(\sin\theta\sin\psi\frac{\partial{p}}{\partial{r}}+\frac{\cos\theta\sin\psi}{r}\frac{\partial{p}}{\partial{\theta}}+\frac{\cos\psi}{r\sin\theta}\frac{\partial{p}}{\partial{\psi}})+ \\
    \frac{m_y}{m_x}(\frac{\alpha}{\alpha_d}\partial_{\mu}p(\sin\theta\cos\psi\frac{\partial{\phi}}{\partial{r}}+\frac{\cos\theta\cos\psi}{r}\frac{\partial{\phi}}{\partial{\theta}}-\frac{\sin\psi}{r\sin\theta}\frac{\partial{\phi}}{\partial{\psi}})) \\
    =F_{V}
    \end{split}
\end{equation}
\begin{equation}
    \begin{split}
    \partial_t{W}+
    m_x(\frac{1}{r^2}\frac{\partial}{\partial{r}}(r^2U\omega)+\frac{1}{r\sin\theta}\frac{\partial}{\partial\theta}(\sin\theta{V\omega}))+\frac{1}{r\sin\theta}\frac{\partial{W\omega}}{\partial{\psi}}+ \\
    -{{m_y}^{-1}}g[\frac{\alpha}{\alpha_{d}}\partial_{\eta}p-\mu_{d}]
    =F_{W}
    \end{split}
\end{equation}
\begin{equation}
    \partial_t{\Theta_m}+{m_x}{m_y}(\frac{1}{r^2}\frac{\partial}{\partial{r}}(r^2U\theta_m)+\frac{1}{r\sin\theta}\frac{\partial}{\partial\theta}(\sin\theta{V\theta_m}))+m_y\frac{1}{r\sin\theta}\frac{\partial{W\theta_m}}{\partial{\psi}}=F_{\Theta_m}
\end{equation}

\end{sloppypar}
\end{document}